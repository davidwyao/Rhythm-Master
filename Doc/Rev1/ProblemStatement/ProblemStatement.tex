\documentclass[12pt,letterpaper]{article}
\usepackage[utf8]{inputenc}
\usepackage[margin=1in]{geometry}
\usepackage[titletoc,title]{appendix}
\usepackage{hyperref}
\usepackage{tabularx}
\usepackage{xcolor}
\usepackage{booktabs}
\usepackage[normalem]{ulem}
\usepackage{indentfirst}

\title{SE 3XA3: Problem Statement\\Frets On Fire}

\author{Team \#16, Rhythm Master
    \\ Almen Ng, nga18
    \\ David Yao, yaod9
    \\ Veerash Palanichamy, palanicv}
    
\date{January 28, 2021}

\begin{document}

\maketitle

\newpage

\tableofcontents
\listoftables
\listoffigures
\newpage

\begin{table}[h!]
\textcolor{red}{\caption{Revision History}}
\begin{tabularx}{\textwidth}{p{3cm}p{2cm}X}
\toprule {\textcolor{red}{\bf Date}} & {\textcolor{red}{\bf Version}} & {\textcolor{red}{\bf Notes}}\\
\midrule
\textcolor{red}{January 26, 2021} & \textcolor{red}{0.1} & \textcolor{red}{Initial Document} \\
\textcolor{red}{January 26, 2021} & \textcolor{red}{0.2} & \textcolor{red}{Completed the introduction of the problem statement} \\
\textcolor{red}{January 27, 2021} & \textcolor{red}{0.5} & \textcolor{red}{Completed the importance and context of the problem statement} \\
\textcolor{red}{April 11, 2021} & \textcolor{red}{1.0} & \textcolor{red}{Added a more explicit explanation of why the problem is important to the stake holder and added a revision history table for Revision 1.0.}\\
\bottomrule
\end{tabularx}
\end{table}

\newpage

\section{Introduction}
Rhythm has been a staple video game genre ever since rhythm games were first invented. They fill a unique niche where music becomes a main focus of gameplay, rather than just background noise. However, well-known rhythm titles such as Rock Band and Guitar Hero are made for gaming consoles, not personal computers. These games often take advantage of specialized input methods, such as guitars and controllers. Frets on Fire is a 2006 open source Python game with gameplay strongly resembling that of Guitar Hero. It is playable with just a keyboard and a computer, making it more accessible than many other titles.

\section{Importance}
Unfortunately, Frets on Fire is built using old libraries and an old version of Python that can no longer be easily compiled using up-to-date software. The game was initially released in 2006 and it has not seen an update since 2008. With that being said, we intend to recreate Frets on Fire with an updated programming language \sout{that would allow users to compile the game with ease}, improved graphics \sout{to make it visually pleasing}, and added test cases that provide full coverage of the game. \textcolor{red}{These are important issues as users should be able to run the program with ease without having to install additional software to run the program. As with the improved graphics, having a visually appealing software game will make the game more immersive and enjoyable for users. Finally, a game with full test coverage will reduce the number of bugs and ensure that the expected outcome is the actual outcome to make the gaming experience better.}

\section{Context}
The primary stakeholders for this game will be video-gamers, users who are interested in practicing music \sout{and artists who want to produce music that can be used in the game}. People of any age can play the game on a desktop or laptop.

Although the game will be built using Unity game engine which uses C\#, the game will be distributed to the users through a standalone binary executable. This executable can be run \textcolor{red}{on} Windows and \textcolor{red}{a} different executable for Linux and Mac will be made available. This makes the game available to a majority of the population as a lot of them have access to a computer with those operating system\textcolor{red}{s}. \\

\end{document}
