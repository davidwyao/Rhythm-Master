\documentclass[12pt,letterpaper]{article}
\usepackage[utf8]{inputenc}
\usepackage[margin=1in]{geometry}
\usepackage[titletoc,title]{appendix}
\usepackage{hyperref}

\title{3XA3 Problem Statement, Revision 0 (Group 16)}
\author{Almen Ng (nga18), David Yao (yaod9), Veerash Palanichamy (palanicv)}
\date{January 28, 2021}

\begin{document}

\maketitle

\section{Introduction}
Rhythm has been a staple video game genre ever since rhythm games were first invented. They fill a unique niche where music becomes a main focus of gameplay, rather than just background noise. However, well-known rhythm titles such as Rock Band and Guitar Hero are made for gaming consoles, not personal computers. These games often take advantage of specialized input methods, such as guitars and controllers. Frets on Fire is a 2006 open source Python game with gameplay strongly resembling that of Guitar Hero. It is playable with just a keyboard and a computer, making it more accessible than many other titles.

\section{Importance}
Unfortunately, Frets on Fire is built using old libraries and an old version of Python that can no longer be easily compiled using up-to-date software. The game was initially released in 2006 and it has not seen an update since 2008. With that being said, we intend to recreate Frets on Fire with an updated programming language that would allow users to compile the game with ease, improved graphics to make it visually pleasing, and added test cases that provide full coverage of the game. 

\section{Context}
The primary stakeholders for this game will be video-gamers, users who are interested in practicing music and artists who want to produce music that can be used in the game. People of any age can play the game on a desktop or laptop. 
\\Although the game will be built using Unity game engine which uses C\#, the game will be distributed to the users through a standalone binary executable. This executable can be run Windows, and  different executable for Linux, and Mac will be made available.   This makes the game available to a majority of the population as a lot of them have access to a computer with those operating system. \\



\end{document}
