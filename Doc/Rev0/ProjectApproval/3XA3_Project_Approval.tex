\documentclass{article}
\usepackage[utf8]{inputenc}
\usepackage[margin=1in]{geometry}
\usepackage[titletoc,title]{appendix}
\usepackage{hyperref}

\title{3XA3 Project Approval (Group 16)}
\author{Almen Ng (nga18), David Yao (yaod9), Veerash Palanichamy (palanicv)}
\date{January 27, 2021}

\begin{document}

\maketitle

\section{Original Project}
Original Project Name: Frets On Fire\\
URL for Project: \url{https://github.com/skyostil/fretsonfire} \\
Number of Lines of Code: 3,000\\
Can the Project be Compiled? Yes.\\
Programming Language: Python 2
\begin{itemize}
    \item Is this programming language feasible for your team? Yes.
    \item Is the domain knowledge understandable within one term? Yes. We aim to get a better understanding throughout the course. 
\end{itemize}

\section{Project Purpose}
Frets on Fire is an open-source rhythm and music video game that essentially involves pressing buttons in time with the coloured markers. The coloured markers appear on the screen and are synchronized with the music that is playing. Points are allocated when a player holds down the corresponding fret buttons on time and correctly. With every ten correct hits, a score multiplier is given. When a player misses or presses a fret button incorrectly, the multiplier resets.

\section{Project Scope}
Recreate the main functionality of the video game Frets on Fire within the context of the 3XA3 course.
The development process will be well-documented and will follow the conventions taught in class. Additionally, we will be programming the game in a modern language, improve on the outdated graphics, as well as provide meaningful test cases. More information included in the "Planned Changes" section.

\section{Hardware Requirements}
The information below was taken directly from the \href{http://fretsonfire.sourceforge.net/about/}{Frets On Fire Website} \\
System Memory 128 MB of RAM \\
Graphics: A fairly fast OpenGL graphics card with decent drivers \\
Windows: Direct X compatible sound card \\ 
Linux: SDL compatible sound card; SDL library installed \\
Mac OS X: Intel processor

\section{Programming Language}
The original code is written in Python 2. We are planning on using C\# to recreate the game. Some of our team members have some basic experience in C\#, and the other members are eager to learn C\#.

\section{Licenses}
\subsection{Bitstream Vera Font License}
This license grants free of charge any person obtaining a copy of the fonts accompanying this license and associated documentation files, "Fonts" and "Font Software" respectively, to reproduce, distribute, copy, merge, publish, and/or sell copies of the the Font Software.

\subsection{GNU General Public License} 
This license ensures freedom of distribution copies of free software, access to source code if needed, changes to the software or use parts of it in new free programs.

\section{More notes on original project}
The original code can be compiled using the appropriate build scripts that is provided. Some test cases that can be used to test the application are playing the game and some input tests to check if the right button was pressed and the accuracy in relation to a predetermined sound (like a beep).

\section{Planned Changes}
Since the original project is coded in Python 2, we plan on programming it in an updated programming language, C\# and using Unity as the game engine. We are planning to simplify this game by choosing one song and develop the game for. We aim to incorporate essential settings and have it run on various operating systems. We also intend to improve the graphics as the graphics seemed outdated as well. There were minimal test cases in the original project so we plan to add more test cases to extend the entire project. 
\end{document}
