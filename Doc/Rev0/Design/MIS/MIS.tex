\documentclass[12pt]{article}
\usepackage[utf8]{inputenc}
\usepackage[margin=1in]{geometry}
\usepackage[titletoc,title]{appendix}
\usepackage{graphicx}
\usepackage{paralist}
\usepackage{amsfonts}
\usepackage{amsmath}
\usepackage{hhline}
\usepackage{booktabs}
\usepackage{multirow}
\usepackage{multicol}
\usepackage{tabularx}
\usepackage[normalem]{ulem}
\usepackage{xcolor}

\pagestyle {plain}
\pagenumbering{arabic}
\setcounter{secnumdepth}{0}

\usepackage{color}

\title{3XA3 Module Interface Specification\\ Rhythm Master} 
\author{Team \#16, Rhythm Masters
    \\ Almen Ng, nga18
    \\ David Yao, yaod9
    \\ Veerash Palanichamy, palanicv}

\date{March 18, 2021}


\begin {document}
 
\maketitle
\newpage
\tableofcontents
\listoftables
\listoffigures

\newpage
\begin{table}[h]
\caption{\bf Revision History}
\begin{tabularx}{\textwidth}{p{3cm}p{2cm}X}
\toprule {\bf Date} & {\bf Version} & {\bf Notes}\\
\midrule
March 1, 2021 & 1.0 & Initial Document\\
March 13, 2021 & 1.1 & Wrote MIS for 10 modules\\
March 17, 2021 & 1.2 & Finished MIS for all modules\\
\bottomrule
\end{tabularx}
\end{table}


\newpage

\section {Button Handler Module}

\subsection{Template Module}
ButtonHandler inherits UnityEngine.MonoBehaviour

\subsection {Uses}
UnityEngine.Input, UnityEngine.GameObject

\subsection {Syntax}

\subsubsection {Exported Constants}
N/A
\subsubsection {Exported Types}
N/A
\subsubsection {Exported Access Programs}

\begin{tabular}{| l | l | l | l |}
\hline
\textbf{Routine name} & \textbf{In} & \textbf{Out} & \textbf{Exceptions}\\
\hline
Start       &           &           &           \\
\hline
Update       &           &           &          \\
\hline
\end{tabular}

\subsection {Semantics}

\subsubsection {State Variables}

\textit{keyToPress:} KeyCode \\
\textit{distance:} $\mathbb{R}$

\subsubsection {Environment Variables}

\textit{button:} Gameplay button that is displayed on screen that can be controlled via the keyboard key that was binded it. It is also a Unity game object.

\subsubsection {State Invariant}
N/A
\subsubsection {Assumptions}
N/A
\subsubsection {Access Routine Semantics}

\noindent Update():
\begin{itemize}
	\item transition: \\
	//\textit{move the button up when the button is pressed, and back when it is released}\\\\
	UnityEngine.Input.GetKeyDown(keyToPress) $\Rightarrow$ \textit{button.z} $:=$ \textit{button.z} $+$ \textit{distance}\\
	UnityEngine.Input.GetKeyUp(keyToPress) $\Rightarrow$ \textit{button.z} $:=$ \textit{button.z} $-$ \textit{distance}
	\item exception: None
\end{itemize}
\subsection{Local Functions/Constants}
N/A

\newpage

\section {Settings Data Module}

\subsection{Template Module}
SettingsData

\subsection {Uses}
UnityEngine.PlayerPrefs

\subsection {Syntax}

\subsubsection {Exported Constants}
N/A
\subsubsection {Exported Types}
N/A
\subsubsection {Exported Access Programs}

\begin{tabular}{| l | l | l | l |}
\hline
\textbf{Routine name} & \textbf{In} & \textbf{Out} & \textbf{Exceptions}\\
\hline
setVolumeLevel    &  $\mathbb{R}$         &           &          \\
\hline
setKeyBinds       &  seq of $\mathbb{N}$     &           &          \\
\hline
getKeyBinds       &           &      seq of $\mathbb{N}$     &          \\
\hline
getVolumeLevel       &           &    $\mathbb{R}$       &           \\
\hline
\end{tabular}

\subsection {Semantics}

\subsubsection {State Variables}
N/A

\subsubsection {Environment Variables}

\subsubsection {State Invariant}
N/A
\subsubsection {Assumptions}
N/A
\subsubsection {Access Routine Semantics}

\noindent setVolumeLevel(v):
\begin{itemize}
	\item transition: UnityEngine.PlayerPrefs.SetFloat(``volume", v)
	\item exception: None
\end{itemize}

\noindent setKeyBinds(s):
\begin{itemize}
	\item transition: $\forall(i: \mathbb{N} | 0 \leq i < |s|: \text{UnityEngine.PlayerPrefs.SetInt}(nameMap[i], s[i]))$
	\item exception: None
\end{itemize}

\noindent getKeyBinds():
\begin{itemize}
	\item output: $out:=\langle i: \mathbb{N} | 0 \leq i < |s|: \text{UnityEngine.PlayerPrefs.GetInt}(nameMap[i], s[i]) \rangle$
	\item exception: None
\end{itemize}

\noindent getVolumeLevel():
\begin{itemize}
	\item output: $out$:=UnityEngine.PlayerPrefs.GetFloat(``volume", v)
	\item exception: None
\end{itemize}
\subsection{Local Functions/Constants}
nameMap: String [``GreenB", ``RedB", ``YellowB", ``BlueB", ``PinkB"]


\newpage

\section {Save File Handler Module}

\subsection{Template Module}
SaveFileHandler

\subsection {Uses}
UnityEngine.PlayerPrefs

\subsection {Syntax}

\subsubsection {Exported Constants}
N/A
\subsubsection {Exported Types}
N/A
\subsubsection {Exported Access Programs}

\begin{tabular}{| l | l | l | l |}
\hline
\textbf{Routine name} & \textbf{In} & \textbf{Out} & \textbf{Exceptions}\\
\hline
writeUserData    &  $\mathbb{N}$, String        &           &          \\
\hline
getAllData    &          &    seq of $\langle$ $\mathbb{N}$, String $\rangle$       &          \\
\hline
\end{tabular}

\subsection {Semantics}

\subsubsection {State Variables}
\textit{filename:} String

\subsubsection {Environment Variables}
\textit{file:} Local file to which user data will be written and read from.

\subsubsection {State Invariant}
N/A
\subsubsection {Assumptions}
N/A
\subsubsection {Access Routine Semantics}

\noindent writeUserData(score, name):
\begin{itemize}
	\item transition: \textit{file} $:=$ \textit{file} $||<$ name score $>$
	\item exception: None
\end{itemize}

\noindent getAllData():
\begin{itemize}
	\item output: $out:= \langle i: \mathbb{N} | 0 \leq i < |file|: \langle file[i][0], file[i][1] \rangle\rangle$\\\textit{//file[x][y] means line x word number y in that file}
	\item exception: None
\end{itemize}

\subsection{Local Functions/Constants}
N/A

\newpage

\section {Settings Menu Module}

\subsection{Template Module} 
SettingsMenu inherits UnityEngine.MonoBehaviour

\subsection {Uses}
SettingsData

\subsection {Syntax}

\subsubsection {Exported Constants}
N/A
\subsubsection {Exported Types}
N/A
\subsubsection {Exported Access Programs}

\begin{tabular}{| l | l | l | l |}
\hline
\textbf{Routine name} & \textbf{In} & \textbf{Out} & \textbf{Exceptions}\\
\hline
Start    &      &           &          \\
\hline
Update   &     &           &          \\
\hline
saveSettings    &          &     &          \\
\hline
sliderUpdate   &          &     &          \\
\hline
\end{tabular}

\subsection {Semantics}

\subsubsection {State Variables}
\textit{temp\_volume:} $\mathbb{N}$\\
\textit{data:} SettingsData

\subsubsection {Environment Variables}
\textit{slider:} Slider used to adjust the volume. The slider can take any value from 0 to 100. This slider will call the \textbf{sliderUpdate()} function when the user changes the slider\\\\
\textit{textfield\_green:} Text field where the user chooses which keyboard key controls the green button during gameplay\\\\
\textit{textfield\_red:} Text field where the user chooses which keyboard key controls the red button during gameplay\\\\
\textit{textfield\_yellow:} Text field where the user chooses which keyboard key controls the yellow button during gameplay\\\\
\textit{textfield\_blue:} Text field where the user chooses which keyboard key controls the blue button during gameplay\\\\
\textit{textfield\_pink:} Text field where the user chooses which keyboard key controls the pink button during gameplay\\\\
\textit{apply\_button:} Button that will trigger the action of changing the actual settings of the game and saving the settings, specifically call the \textbf{saveSettings()} function.
\subsubsection {State Invariant}
N/A
\subsubsection {Assumptions}
N/A
\subsubsection {Access Routine Semantics}

\noindent Start():
\begin{itemize}
	\item transition: \\
	value of \textit{slider} $:=$ \textit{data}.getVolumeLevel()\\
	value of \textit{textfield\_green} $:=$ \textit{data}.getKeyBinds()[0]\\
	value of \textit{textfield\_red} $:=$ \textit{data}.getKeyBinds()[1]\\
	value of \textit{textfield\_yellow} $:=$ \textit{data}.getKeyBinds()[2]\\
	value of \textit{textfield\_blue} $:=$ \textit{data}.getKeyBinds()[3]\\
	value of \textit{textfield\_pink} $:=$ \textit{data}.getKeyBinds()[4]\\
	\item exception: None
\end{itemize}

\noindent saveSettings():
\begin{itemize}
	\item transition: \\
	\textit{data}.setVolumeLevel(temp\_volume)\\
	\textit{data}.setKeyBinds($\langle$ value(\textit{textfield\_green}), \\
	\hspace*{3.65cm}value(\textit{textfield\_red}), \\
	\hspace*{3.65cm}value(\textit{textfield\_yellow}), \\
	\hspace*{3.65cm}value(\textit{textfield\_blue}), \\
	\hspace*{3.65cm}value(\textit{textfield\_pink})$\rangle$)
	
	\item exception: None
\end{itemize}

\newpage

\noindent sliderUpdate():
\begin{itemize}
	\item transition: \textit{temp\_volume} $:=$ slider\_value(\textit{slider})
	\item exception: None
\end{itemize}

\subsection{Local Functions/Constants}

value: \textit{textfield}$\rightarrow \mathbb{N}$\\
value $\equiv$ returns the value that the textfields contains\\

\noindent slider\_value: \textit{slider}$\rightarrow \mathbb{N}$\\
slider\_value $\equiv$ returns the value of a slider

\newpage

\section {Leaderboard Module}

\subsection{Template Module}
Leaderboard inherits UnityEngine.MonoBehaviour

\subsection {Uses}
SaveFileHandler, LeaderboardCalculator

\subsection {Syntax}

\subsubsection {Exported Constants}
N/A
\subsubsection {Exported Types}
N/A
\subsubsection {Exported Access Programs}

\begin{tabular}{| l | l | l | l |}
\hline
\textbf{Routine name} & \textbf{In} & \textbf{Out} & \textbf{Exceptions}\\
\hline
Start    &      &           &          \\
\hline
Update   &      &           &          \\
\hline
\end{tabular}

\subsection {Semantics}

\subsubsection {State Variables}
\textit{saveFile:} SaveFileHandler
\textit{playerList:} seq of $\langle String, \mathbb{N} \rangle$

\subsubsection {Environment Variables}
\textit{table:} table that displays the player rank,  name and score. This table's data can be indexed such as table[row][column] where 1st columns is the rank, the 2nd the name and lastly the score. The table will also have a heading of rank, player and score.

\subsubsection {State Invariant}
N/A
\subsubsection {Assumptions}
\textit{saveFile} should have been assigned referenced to a SaveFileHandler component
\subsubsection {Access Routine Semantics}

\noindent Start():
\begin{itemize}
	\item transition: \\
	$\forall(i: \mathbb{N}| 0 \le i < |s|:table[i][0]=i+1 \land table[i][1]=s[i][1] \land  table[i][2]=s[i][2])$\\ 
	where $s=$ LeaderboardCalculator.Sort($saveFile.$getAllData())
	\item exception: None
\end{itemize}

\subsection{Local Functions/Constants}
N/A

\newpage

\section{Game Manager}

\subsection{Template Module}
GameManager inherits UnityEngine.MonoBehaviour

\subsection{Uses}
System.Collections, Systems.Collections.Generic, UnityEngine, UnityEngine.UI

\subsection{Syntax}
\subsubsection{Exported Constants}
N/A

\subsubsection{Exported Types}
N/A

\subsubsection{Exported Access Programs}
\begin{tabular}{| l | l | l | l |}
\hline
\textbf{Routine name} & \textbf{In} & \textbf{Out} & \textbf{Exceptions}\\
\hline
Start    &      &           &          \\
\hline
Update   &     &           &          \\
\hline
\end{tabular}

\subsection{Semantics}
\subsubsection{State Variables}
\textit{startPlaying:} boolean\\
\textit{music:} AudioSource\\
\textit{instance:} GameManager

\subsubsection{Environment Variables}
\textit{theNS:} NoteScroller controlling the movement of notes along the game screen.

\subsubsection{State Invariant}
N/A

\subsubsection{Assumptions}
N/A

\subsubsection{Access Routine Semantics}
\noindent Start():
\begin{itemize}
	\item transition:\\
	    instance $:=$ this
	\item exception: None
\end{itemize}
\noindent Update():
\begin{itemize}
	\item transition:\\
	    Initiates both music playing and note scrolling when startPlaying = true.
	\item exception: None
\end{itemize}

\subsection{Local Functions/Constants}
N/A

\newpage


\section{Effects}

\subsection{Template Module}
Effects inherits UnityEngine.MonoBehaviour

\subsection{Uses}
System.Collections, System.Collections.Generic, UnityEngine

\subsection{Syntax}
\subsubsection{Exported Constants}
N/A

\subsubsection{Exported Types}
N/A

\subsubsection{Exported Access Programs}
\begin{tabular}{| l | l | l | l |}
\hline
\textbf{Routine name} & \textbf{In} & \textbf{Out} & \textbf{Exceptions}\\
\hline
Start    &      &           &          \\
\hline
Update   &     &           &          \\
\hline
\end{tabular}

\subsection{Semantics}
\subsubsection{State Variables}
\textit{lifetime:} $\mathbb{R}$\\
\textit{missEffect:} GameObject\\
\textit{okEffect:} GameObject\\
\textit{goodEffect:} GameObject\\
\textit{perfEffect:} GameObject

\subsubsection{Environment Variables}
N/A

\subsubsection{State Invariant}
\textit{lifetime} $\geq 0$

\subsubsection{Assumptions}
N/A

\subsubsection{Access Routine Semantics}
\noindent Start():
\begin{itemize}
	\item transition: Display either \textit{missEffect}, \textit{okEffect}, \textit{goodEffect}, \textit{perfEffect}.
	\item exception: None
\end{itemize}
\noindent Update():
\begin{itemize}
	\item transition:\\
	    Effect is deleted after \textit{lifetime} has passed.
	\item exception: None
\end{itemize}

\subsection{Local Functions/Constants}
N/A

\newpage

\section{Leaderboard Calculator}

\subsection{Template Module}
LeaderboardCalculator inherits UnityEngine.MonoBehaviour

\subsection{Uses}
System.Collections.Generic

\subsection{Syntax}
\subsubsection{Exported Constants}
N/A

\subsubsection{Exported Types}
\textit{playerList:} seq of $\langle String, \mathbb{N} \rangle$

\subsubsection{Exported Access Programs}
\begin{tabular}{| l | l | l | l |}
\hline
\textbf{Routine name} & \textbf{In} & \textbf{Out} & \textbf{Exceptions}\\
\hline
Sort & seq of $\langle String, \mathbb{N} \rangle$ & seq of $\langle String, \mathbb{N} \rangle$ & None\\
\hline
\end{tabular}

\subsection{Semantics}
\subsubsection{State Variables}
\textit{playerList:} seq of $\langle String, \mathbb{N} \rangle$

\subsubsection{Environment Variables}
N/A

\subsubsection{State Invariant}
N/A

\subsubsection{Assumptions}
The scores in playerList are all from the same game track.

\subsubsection{Access Routine Semantics}
\noindent Sort(vector<pair<string,int>> playerList):
\begin{itemize}
	\item transition:\\
	    Values in playerList are sorted in descending order of their int values.
	\item exception: None
\end{itemize}

\subsection{Local Functions/Constants}
None

\newpage

\section{Instructions}

\subsection{Template Module}
Instructions inherits UnityEngine.MonoBehaviour

\subsection{Uses}
UnityEngine.Input, UnityEngine.GameObject

\subsection{Syntax}
\subsubsection{Exported Constants}
N/A
\subsubsection{Exported Types}
N/A
\subsubsection{Exported Access Programs}
\begin{tabular}{| l | l | l | l |}
\hline
\textbf{Routine name} & \textbf{In} & \textbf{Out} & \textbf{Exceptions}\\
\hline
Toggle   &     &           &          \\
\hline
\end{tabular}

\subsection{Semantics}
\subsubsection{State Variables}
N/A

\subsubsection{Environment Variables}
\textit{instructionUI:} GameObject showing the instructions\\
\textit{toggleButton:} Button to toggle the instruction screen

\subsubsection{State Invariant}
N/A

\subsubsection{Assumptions}
N/A

\subsubsection{Access Routine Semantics}
\noindent Toggle():
\begin{itemize}
	\item transition:\\
	    Displays the instruction screen if it is not currently being displayed, otherwise stops displaying it.
	\item exception: None
\end{itemize}

\subsection{Local Functions/Constants}
N/A

\newpage

\section{Note Spawner}

\subsection{Template Module}
NoteSpawner inherits UnityEngine.MonoBehaviour

\subsection{Uses}
UnityEngine

\subsection{Syntax}
\subsubsection{Exported Constants}
N/A
\subsubsection{Exported Types}
N/A
\subsubsection{Exported Access Programs}
\begin{tabular}{| l | l | l | l |}
\hline
\textbf{Routine name} & \textbf{In} & \textbf{Out} & \textbf{Exceptions}\\
\hline
Start   &     &           &          \\
SpawnNote   &     &           &     \\
\hline
\end{tabular}

\subsection{Semantics}
\subsubsection{State Variables}
\textit{spawnDelay:} $\mathbb{R}$
\textit{spawnStartTime:} $\mathbb{R}$
\textit{note:} GameObject

\subsubsection{Environment Variables}
N/A

\subsubsection{State Invariant}
N/A

\subsubsection{Assumptions}
N/A

\subsubsection{Access Routine Semantics}
\noindent Start():
\begin{itemize}
	\item transition:\\
	    Repeatedly calls SpawnNote, starting after \textit{spawnStartTime} and repeating every \textit{spawnDelay}.
	\item exception: None
\end{itemize}
\noindent SpawnNote():
\begin{itemize}
	\item transition:\\
	    Instantiates a single \textit{note} and adds it to the game.
	\item exception: None
\end{itemize}

\subsection{Local Functions/Constants}
N/A

\newpage

\section{Pause Menu}

\subsection{Template Module}
PauseMenu inherits UnityEngine.MonoBehaviour

\subsection {Uses}
UnityEngine.Input, UnityEngine.GameObject

\subsection {Syntax}

\subsubsection {Exported Constants}
N/A
\subsubsection {Exported Types}
N/A

\subsubsection {Exported Access Programs}

\begin{tabular}{| l | l | l | l |}
\hline
\textbf{Routine name} & \textbf{In} & \textbf{Out} & \textbf{Exceptions}\\
\hline
Update & & & \\
\hline
Resume & & & \\
\hline
Pause & & & \\
\hline 
\end{tabular}

\subsection {Semantics}

\subsubsection {State Variables}
\textit{gamePaused}: $\mathbb{B}$ \\

\subsubsection {Environment Variables}
\textit{pauseMenuUI}: GameObject that shows the pause menu

\subsubsection {State Invariant}
None

\subsubsection {Assumptions}
The environment variables are initialized manually through the Unity interface.

\subsubsection {Access Routine Semantics}
Update():
\begin{itemize}
    \item Transition: Checks if "esc" button has been pressed and \textit{gamePaused} is True. If both are True, call \textbf{Resume()}, otherwise, if only "esc" button has been pressed, call \textbf{Pause()}.
    \item Exception: None
\end{itemize}


\noindent Resume():
\begin{itemize}
    \item Transition: Set \textit{pauseMenuUI} to false, unfreeze time, and set \textit{gamePaused} to True.
    \item Exception: None
\end{itemize}

\noindent Pause():
\begin{itemize}
    \item Transition: Set \textit{pauseMenuUI} to active, freeze time, and set \textit{gamePaused} to True.
    \item Exception: None
\end{itemize}


\subsection{Local Functions/Constants}
N/A
\medskip


\newpage

\section{Main Menu}

\subsection{Template Module}
MainMenu inherits UnityEngine.MonoBehaviour

\subsection {Uses}
UnityEngine.Input, UnityEngine.GameObject, UnityEngine.SceneManagement

\subsection {Syntax}

\subsubsection {Exported Constants}
N/A
\subsubsection {Exported Types}
N/A

\subsubsection {Exported Access Programs}

\begin{tabular}{| l | l | l | l |}
\hline
\textbf{Routine name} & \textbf{In} & \textbf{Out} & \textbf{Exceptions}\\
\hline
PlayGame & & & \\
\hline
QuitGame & & & \\
\hline
NavigateSettings & & & \\
\hline
\end{tabular}

\subsection {Semantics}

\subsubsection {State Variables}
N/A

\subsubsection {Environment Variables}
\textit{gameUI}: Unity Scene for the game play\\

\noindent \textit{settingsUI}: Unity Scene that enables editing of the settings\\

\noindent \textit{startGameButton}: Button that will trigger the action of navigating to the \textit{gameUI}, specifically calling  the \textbf{PlayGame()} function.\\

\noindent \textit{quitGameButton}: Button that will trigger the action of quitting the game, specifically calling the \textbf{QuitGame()} function.\\

\noindent \textit{navigateSettingsButton}: Button that will trigger the action of navigating to the \textbf{Settings Menu}, specifically calling the \textbf{NavigateSettings()} function.\\

\subsubsection {State Invariant}

\subsubsection {Assumptions}
The environment variables are initialized manually through the Unity interface.

\subsubsection {Access Routine Semantics}
PlayGame():
\begin{itemize}
    \item Transition: Navigates to \textit{gameUI} once \textit{startGameButton} is pressed.
    \item Exception: None
\end{itemize}


\noindent QuitGame():
\begin{itemize}
    \item Transition: Quits the application once \textit{quitGameButton} is pressed.
    \item Exception: None
\end{itemize}

\noindent NavigateSettings():
\begin{itemize}
    \item Transition: Navigates to \textit{settingsUI} once \textit{navigateSettingsButton} is pressed.
    \item Exception: None
\end{itemize}

\subsection{Local Functions/Constants}
N/A
\medskip


\newpage

\section{Collision Detector}

\subsection{Template Module}
Collision Detector inherits  UnityEngine.MonoBehavior

\subsection {Uses}
UnityEngine.GameObject, UnityEngine.Collider

\subsection {Syntax}

\subsubsection {Exported Constants}
N/A
\subsubsection {Exported Types}
N/A

\subsubsection {Exported Access Programs}

\begin{tabular}{| l | l | l | l |}
\hline
\textbf{Routine name} & \textbf{In} & \textbf{Out} & \textbf{Exceptions}\\
\hline
OnTriggerEnter & Collider & &\\
\hline
OnTriggerExit & Collider & &\\
\hline
\end{tabular}

\subsection {Semantics}

\subsubsection {State Variables}
\textit{canBePressed}: $\mathbb{B}$ \\

\subsubsection {Environment Variables}
\textit{noteCollider}: A ColliderObject repersenting a note

\subsubsection {State Invariant}
N/A

\subsubsection {Assumptions}
The environment variables are initialized manually through the Unity interface.

\subsubsection {Access Routine Semantics}
\noindent OnTriggerEnter(noteCollider)
\begin{itemize}
    \item Transition: Checks if \textit{noteCollider} is an "Activator". If it is, \textit{canBePressed} will be set to True.
    \item Exception: None
\end{itemize}

\noindent OnTriggerExit(noteCollider):
\begin{itemize}
    \item Transition: Checks if \textit{noteCollider} is an "Activator". If it is, \textit{canBePressed} will be set to Falsed.
    \item Exception: None
\end{itemize}

\subsection{Local Functions/Constants}
N/A
\medskip


\newpage

\section{Note Scroller}

\subsection{Module}
N/A

\subsection {Uses}
UnityEngine.GameObject

\subsection {Syntax}

\subsubsection {Exported Constants}
N/A
\subsubsection {Exported Types}
N/A

\subsubsection {Exported Access Programs}

\begin{tabular}{| l | l | l | l |}
\hline
\textbf{Routine name} & \textbf{In} & \textbf{Out} & \textbf{Exceptions}\\
\hline
Start & & & \\
\hline
Update & & & \\
\hline
\end{tabular}

\subsection {Semantics}

\subsubsection {State Variables}
\textit{hasStarted}: $\mathbb{B}$ \\

\subsubsection {Environment Variables}
N/A

\subsubsection {State Invariant}
N/A

\subsubsection {Assumptions}
N/A

\subsubsection {Access Routine Semantics}
\noindent Start():
\begin{itemize}
    \item Transition: Sets \textit{beatTempo}
    \item Exception: None
\end{itemize}

\noindent Update():
\begin{itemize}
    \item Transition: Checks \textit{hasStarted}. If True, move the GameObject based on \textit{beatTempo}. 
    \item Exception: None
\end{itemize}


\subsection{Local Functions/Constants}
\textit{beatTempo}: $\mathbb{N} $ \\
\textit{beatTempo} $\equiv$ 126
\medskip


\newpage

\section{Score Calculator}

\subsection{Template Module}
ScoreCalculator

\subsection {Uses}
N/A

\subsection {Syntax}

\subsubsection {Exported Constants}
N/A

\subsubsection {Exported Types}
N/A

\subsubsection {Exported Access Programs}

\begin{tabular}{| l | l | l | l |}
\hline
\textbf{Routine name} & \textbf{In} & \textbf{Out} & \textbf{Exceptions}\\
\hline
Start & & &\\
\hline
NoteHit & & & \\
\hline
NormalHit & & & \\
\hline
GoodHit & & & \\
\hline
PerfectHit & & & \\
\hline
NoteMissed & & & \\
\hline
\end{tabular}

\subsection {Semantics}

\subsubsection {State Variables}
\textit{currentMultiplier}: $\mathbb{N}$\\
\textit{multiplierTracker}: $\mathbb{N} $ \\
\textit{multiplierThresholds}: seq of $<\mathbb{N}>$ \\
\textit{totalNotes}: $\mathbb{N} $ \\
\textit{normalHits}: $\mathbb{N} $ \\
\textit{goodHits}: $\mathbb{N} $ \\
\textit{perfectHits}: $\mathbb{N} $ \\
\textit{missedHits}: $\mathbb{N} $ \\
\textit{currentScore}: $\mathbb{N} $ \\
\textit{totalNotes}: $\mathbb{N}$

\subsubsection {Environment Variables}
N/A

\subsubsection {State Invariant}
N/A

\subsubsection {Assumptions}
\textit{multiplierThresholds} are set manually within the Unity interface.

\subsubsection {Access Routine Semantics}
\noindent Start():
\begin{itemize}
    \item Transition: Initializes \textit{currentScore} to 0, \textit{currentMultiplier} to 1, and \textit{multiplierTracker} to 0. Sets \textit{totalNotes} to the number of Note GameObjects.
    \item Exception: None
\end{itemize}

\noindent NoteHit():
\begin{itemize}
    \item Transition: Increments \textit{multiplierTracker} every time a note is hit consecutively and increments the \textit{currentMultiplier} once a threshold based on \textit{multiplierThresholds} is met by comparing \textit{multiplierThreshold}[\textit{currentMultiplier}] and \textit{multiplierTracker}.
    \item Exception: None
\end{itemize}

\noindent NormalHit():
\begin{itemize}
    \item Transition: Updates \textit{currentScore} by adding \textit{scorePerNote} multiplied by \textit{currentMultiplier}. Increments \textit{normalHits} each time this function is called.
    \item Exception: None
\end{itemize}

\noindent GoodHit():
\begin{itemize}
    \item Transition: Updates \textit{currentScore} by adding \textit{scorePerGoodNote} multiplied by \textit{currentMultiplier}. Increments \textit{GoodHits} each time this function is called.
    \item Exception: None
\end{itemize}

\noindent PerfectHit():
\begin{itemize}
    \item Transition: Updates \textit{currentScore} by adding \textit{scorePerPerfectNote} multiplied by \textit{currentMultiplier}. Increments \textit{GoodHits} each time this function is called.
    \item Exception: None
\end{itemize}

\noindent NoteMissed():
\begin{itemize}
    \item Transition: Resets the \textit{CurrentMultiplier} and \textit{multiplierTracker} to its initial values, 1 and 0 respectively. Increments \textit{missedHits} by one each this this function is called. 
    \item Exception: None
\end{itemize}

\subsection{Local Functions/Constants}
\textit{scorePerNote}: $\mathbb{N} $ \\
\textit{scorePerNote} $\equiv$ 100\\

\noindent \textit{scorePerGoodNote}: $\mathbb{N} $ \\
\textit{scorePerGoodNote} $\equiv$ 125\\

\noindent \textit{scorePerPerfectNote}: $\mathbb{N} $ \\
\textit{scorePerPerfectNote} $\equiv$ 150\\
\medskip
\end {document}
